% spellchecker: disable

\documentclass[12pt]{article}

\usepackage[margin=1in]{geometry}
\usepackage{amsmath}
% \usepackage{amssymb}
% \usepackage{mathtools}
% \usepackage{color}
\usepackage{enumitem}
% \usepackage{makecell}
% \usepackage{hyperref}

\setlist{parsep=0em, itemsep=0.5em}
% \setlength{\parskip}{0em}
% \setlength\parindent{0pt}

% \renewcommand{\theenumi}{\alph{enumi}}

\title{Fiber Generator Controlled Parameters}
\author{Matthew Dutson}

% spellchecker: enable

\begin{document}

\maketitle
\thispagestyle{empty}
\pagestyle{empty}

    
\section*{Definitions}


\subsection*{Fiber Count}

The is is the integer number $m$ of fibers in the image.


\subsection*{Mean Fiber Length}

The path of a fiber is defined by a tuple $P = (\vec{p}_1, \vec{p}_2, \ldots, \vec{p}_n)$. The fiber length $l$ is given by
\begin{equation*}
    l = \sum_{i = 1}^{n - 1} |\vec{p}_{i + 1} - \vec{p}_i|.
\end{equation*}
If we assume that all segments have equal length $l_s$, then the fiber length is simply
\begin{equation*}
    l = (n - 1) \cdot l_s.
\end{equation*}


\subsection*{Mean Fiber Width}

The varying width is given by a tuple $W$ with $n - 1$ elements. $w_j$ gives the width of the fiber between points $\vec{p}_j$ and $\vec{p}_{j + 1}$. Because all segments have the same length, the average for a fiber is
\begin{equation*}
    \bar{w} = \frac{1}{n - 1} \sum_{j = 1}^{n - 1} w_j
\end{equation*}
The image-wide average is the average over fibers weighted by the length of each fiber. Let there be $m$ fibers, each with width tuple $W_i$ and number of segments $n_i$.
\begin{equation*}
    \bar{w}_\text{tot} = 
        \frac{1}{\sum_{i = 1}^m n_i} 
        \sum_{i = 1}^m \sum_{j = 1}^{n_i - 1} w_{i,j}
\end{equation*}
This is equivalent to taking the fiber-independent mean of segment widths.


\subsection*{Mean Fiber Angle}

The angle of the $i^\text{th}$ fiber is simply the angle between its endpoints, $\vec{p}_{i,1}$ and $\vec{p}_{i,n}$.
\begin{equation*}
    \theta_i = \cos^{-1} \left(\frac{(\vec{p}_{i,n} - \vec{p}_{i,1})_x}{|\vec{p}_{i,n} - \vec{p}_{i,1}|}\right)
\end{equation*}
A normalized ``mean vector'' is found by multiplying all fiber angles by 2 and taking the normalized, length-weighted mean of the resultant direction vectors.
\begin{equation*}
    \bar{\theta} = \frac{1}{2} \cdot \text{arg} \left(
    \frac{1}{\sum_{i = 1}^m n_i} 
    \sum_{i = 1}^m n_i \cdot e^{2 i \cdot \theta_i} \right)
\end{equation*}


\subsection*{Overall Fiber Alignment}

The alignment $A$ is the magnitude of the ``mean vector'' from the previous part.
\begin{equation*}
    A = \text{mag} \left(
    \frac{1}{\sum_{i = 1}^m n_i} 
    \sum_{i = 1}^m n_i \cdot e^{2 i \cdot \theta_i} \right)
\end{equation*}


\subsection*{Mean Fiber Straightness}

The straightness of an individual fiber is the ratio of its length to the distance between its endpoints. Let $l_i$ be the length of the $i^\text{th}$ fiber and assume it contains $n$ points. Then the straightness $s_i$ is
\begin{equation*}
s_i = \frac{|\vec{p}_n - \vec{p}_1|}{l_i}.
\end{equation*}
The image-wide mean is weighted by fiber length.
\begin{equation*}
\bar{s} = 
    \frac{1}{\sum_{i = 1}^m n_i}
    \sum_{i = 1}^m n_i \cdot s_i
\end{equation*}


\section*{Parameter Generation}


\section*{Output}

For each generated fiber, the following details will be recorded as output:
\begin{itemize}
    \item The tuple $P = (\vec{p}_1, \vec{p}_2, \ldots, \vec{p}_n)$ of points which define the path of the fiber.
    \item The tuple $W = (\vec{w}_i, \vec{w}_2, \ldots, \vec{w}_{n - 1})$ of widths.
\end{itemize}
Other details (local density, individual fiber angle, etc.) from these tuples.

\end{document}